
\section{Repr�sentation complexe des grandeurs sinuso�dales}

\section{Imp�dances complexes}

L'imp�dance est d�finie comme le rapport de la tension $u(t)$ (en V) et du courant $i(t)$ o� $u(t)=U_0 e^{j\omega t+\phi_i}$ avec $U_0$ la tension max et $\omega$ la pulsation du signal, et o� $u(t)=I_0 e^{j\omega t+\phi_u}$ avec $I_0$ est l'intensit� max. On �crit :

\begin{equation}\label{imp_comp}
 Z=\frac{u}{i}
\end{equation}

\subsection{R�sistance}

Soit $Z_R$ l'imp�dance complexe d'une r�sistance. D'apr�s l'�quation \ref{imp_comp}, $Z_R$ est la r�sistance r�elle du dip�le telle que:

\begin{equation}
 Z_R=\frac{U_0}{I_0}=R
\end{equation}

\subsection{Capacitance}

Soit $Z_C$ l'imp�dance complexe d'une capacitance (condensateur). On sait que $i(t)=C \frac{\partial u(t)}{\partial t}$. Or $\frac{\partial u(t)}{\partial t} = j\omega U_0 e^{j\omega t}$. On en d�duit d'apr�s l'�quation \ref{imp_comp}:

 \begin{equation}
  Z_C=\frac{1}{j C\omega}
 \end{equation}

\subsection{Inductance}

Soit $Z_L$ l'imp�dance complexe d'une inductance (sol�no�de). On sait que $u(t)=L \frac{\partial i(t)}{\partial t}$. Or $\frac{\partial i(t)}{\partial t} = j\omega I_0 e^{j\omega t}$. On en d�duit d'apr�s l'�quation \ref{imp_comp}:

 \begin{equation}
  Z_L=j L\omega
 \end{equation}

\subsection{Puissances}

\subsubsection{Puissance instantan�e}

\begin{equation}
 p=ui\ \ \ (\mathrm{W = J.s^{-1}})
\end{equation}

\subsubsection{Puissance active (absorb�e)}

La puissance active correspond � l'�nergie moyenne consomm�e par unit� de temps \cad � la \textbf{puissance moyenne consomm�e} :

\begin{equation}
 P=<p>=\frac{1}{T}\int_0^T p\ dt
\end{equation}

d'o�, en r�gime sinuso�dal:

\begin{equation}
 P=\frac{U_0 I_0}{2} \cos \phi
\end{equation}

soit,

\begin{equation}
 P=U I \cos \phi
\end{equation}

o� $I=\frac{I_0}{\sqrt{2}}$ et $U=\frac{U_0}{\sqrt{2}}$ sont respectivement la valeur efficace du courant et la valeur efficace de la tension dans le circuit.

\subsubsection{Puissance r�active}

\begin{equation}
 Q=\frac{U_0 I_0}{2} \sin \phi = U I \sin \phi
\end{equation}

\subsubsection{Puissance apparente}

\begin{eqnarray}
 S & = & U I\\
 S & = & \sqrt{P^2 + Q^2}
\end{eqnarray}


\section{Etude de cas}

\subsection{Circuit RLC}

\subsection{Th�or�me de Th�venin en notation complexe}

\subsection{Th�or�me de Norton en notation complexe}

